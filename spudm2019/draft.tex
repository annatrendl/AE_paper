\documentclass[12pt, a4paper]{article}
\usepackage{color}

\begin{document}

\textcolor{red}{\textbf{A long abstract of maximum 4500 characters including spaces, specifying the aim, method, results and conclusions of the research.}}

The attraction effect is one of the most widely researched cognitive biases in decision making research. Its importance stems from the fact that it poses a challenge to all choice models that rely on the assumption that preferences can be represented on a cardinal utility scale. Consequently, the attraction effect has played a substantial role in the evolution of multialternative, multiattribute models of choice over the past 20 years.



In previous investigations, the attraction effect has almost exclusively only been demonstrated in settings where the options were presented with numeric attributes. This somewhat abstract presentation format is substantially different from the overwhelming majority of choices people face in their everyday life, which most often involve complex, naturalistic options. Frederick, Lee, and Baskin (2014) and Yang and Lynn (2014) have presented dozens of experiments where they tested the attraction effect using real-world stimuli, and reported no attraction effect in any of these, which elicited several responses, sparking a debate in the literature. In one of these reactions, Huber, Payne, and Puto (2014) criticised the methodology used by Frederick et al., and laid down a set of criteria that should be met by any experiment wishing to test for the attraction effect in real-world consumer choices. While this debate has seemingly ended, we believe that the literature is still lacking a conclusive answer regarding the real-world relevance of the attraction effect. Knowing whether this influential decision bias is only confined to cases where the options have a specific, artificial presentation format is crucial information for the development of formal models of choice. Therefore, in this paper, we present the first rigorous test of the attraction effect using real-world stimuli.



In our experiment, participants were presented with attraction effect choice triplets created from movie posters. We created bespoke triplets for each participant, using genre information and the participant’s preferences over the movies (obtained in a rating stage that preceded the choice stage). To make sure that these choice triplets indeed constituted an attraction effect choice situation for every participant, we also collected information on the perceived similarity of the target-competitor and target-decoy pairs. As we demonstrate, our results strongly suggest that we met all the criteria set out by Huber et al. (2014) for a stringent test of the attraction effect. Moreover, we have utilised a strict exclusion criteria to make sure we only include participants who took the task sufficiently seriously. The method, exclusion criteria, and analyses conducted were all registered before any data was collected.
Our results unequivocally suggest that participants were perfectly indifferent between the target and the competitor. In other words, the presence of the decoy did not seem to alter preferences in this choice task, and we did not observe the attraction effect. Perhaps surprisingly, the perceived similarity of the target-decoy and target-competitor pairs, familiarity with the movies, and relative preference between the target and the decoy did not seem to modulate the strength of the attraction effect either.


To conclude, our findings are strongly in line with that of Frederick et al. (2014) and Yang and Lynn (2014). However, the importance of our results stems from our methodological approach: we aimed to create choice scenarios identical to those seen in experiments that have previously found a strong attraction effect, the only difference being the type of the stimuli – naturalistic as opposed to the standard numerical format. As such, we ensured the best possible circumstances for the attraction effect to arise, yet we still found a clear null effect. Therefore, our findings constitute a much stronger basis for the claim that the attraction effect does indeed not extend to choices involving more complex, naturalistic options. While we do not speculate about the underlying reasons for this phenomenon, our results raise serious concerns about the role the attraction effect should play in choice modelling research.



\end{document}