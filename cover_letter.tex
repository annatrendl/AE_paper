\documentclass{letter}
\usepackage[margin=1.5in]{geometry}
\signature{Anna Trendl}
\address{Department of Psychology\\ University of Warwick \\ Gibbet Hill Road, Coventry \\ CV4 7AL, UK}
\begin{document}

\begin{letter}
{}
\opening{Dear X,}


Please consider our manuscript "A zero attraction effect in naturalistic choice" for publication in the Journal of Marketing Research. In this article, we describe what we believe to be the first truly rigorous experimental investigation of the attraction effect using complex, real-world stimuli.

Whether the attraction effect extends to choices that involve options with naturalistic attributes is a key issue for decision-making research. One of the most significant implications of the attraction effect is that the utility of any option cannot be represented on a cardinal scale that is stable across choice sets. This bears substantial importance for choice modelling. Accordingly, the attraction effect is one of the most widely researched decision biases, and it has contributed to the development of a wide variety of highly influential choice models. If this cognitive bias was confined to choices where options have numerical attributes, that would cast serious doubt on the real world validity and usefulness of models that were developed using insights from attraction effect research.

Arguably, the overwhelming majority of choice experiments that demonstrated the attraction effect have used stimuli presented in a numerical attribute-by-alternative format. Frederick, Lee, and Baskin (2014) have presented a series of experiments that aimed to test the attraction effect using a wide range of naturalistic choice options, and reported no evidence for the attraction effect. In a response article, Huber, Payne, and Puto (2014) have pointed out several shortcomings of Frederick et al.'s experimental approach, claiming that their experiments do not meet the necessary conditions for conducting a rigorous test of the attraction effect. Consequently, the literature is still lacking a conclusive answer to the question whether the attraction effect extends to choices that involve naturalistic options. This paper provides this rigorous test that is missing form the literature.

Drawing on insights from this debate, our article describes a carefully designed experiment that tests the attraction effect with real-world choice options, while addressing all the methodological shortcomings of previous investigations. In line with Frederick et al.'s results, we find a precisely zero attraction effect.


Thank you for your time and consideration.


\closing{Sincerely,}



\end{letter}
\end{document}
